\documentclass{article}

\title{instruction timeline}
\author{erin marshall \texttt{<limarshall@wisc.edu>}}

% 1 2 3 4 5
% F D X M W

% % lbi r0, 0 
% % lbi r5, 43
% % lbi r6, 43
% % lbi r7, 43        
% % ld r1, r0, 0
% % st r5, r1, 0
% % ld r1, r0, 2
% % st r6, r1, 1
% % ld r1, r0, 4
% % st r7, r1, 1
% % halt

\begin{document}

\maketitle
Instead of an instruction timeline, I explain the hardware as implemented (as per Rutwik Jain's answer on Piazza to \texttt{@385}).

I put significant effort into avoiding any potential stalls. As such, the pipeline could be described as \textit{fully-forwarded};
only memory loads incur a stall (when EX needs a result from dmem that has not completed the MEM stage yet). Namely, the following paths
are implemented:

\begin{enumerate}
    \item ALU output latched at EX/MEM forwarded to either ALU port in EX
    \item ALU output latched at MEM/WB forwarded to either ALU port in EX (previous MEM\(\,\rightarrow\,\)EX takes priority over this)
    \item writeback data latched at MEM/WB forwarded to MEM dmem write input if WB of the older instruction is writing a register we're writing to dmem
\end{enumerate}

These paths are integrated into their respective stages, and managed by a top-level unit called the \texttt{forwarder}, which observes all relevant signals and sends out mux controls.

One data hazard remains: when an instruction reading from dmem is immediately followed by one needing that dmem result in its EX stage. We must stall the second instruction by one cycle
to properly await the dmem result; this is done by the \texttt{hazard} unit, used similarly to the previous \texttt{forwarder}.

Control hazards are managed by the \texttt{branch\_controller}, which decides when and how to resteer mispredicted branches. Note that displaced unconditional jumps,
where both the target and the direction are known in ID, are subject to an "early resteer". Early resteers rewrite the PC and flush IF/ID immediately following decoding,
rather than a typical computed jump or conditional branch which requires a late resteer as part of the EX stage of branch execution.

\end{document}